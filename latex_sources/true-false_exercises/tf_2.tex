
\documentclass[letterpaper]{article}

\usepackage{amsmath}
\usepackage{etoolbox}
\usepackage{amsfonts}
\usepackage{amssymb}
\usepackage{rotating}
\usepackage[tracking=true]{microtype}
\usepackage{booktabs}
\usepackage[nolinks]{qrcode}

\usepackage[ linkcolor=black, citecolor=linkcol, urlcolor= black, colorlinks=true, hypertexnames=false]
{hyperref}


%---------------------------------------------------
% Page geometry
%---------------------------------------------------
\usepackage[
                      top    = 1in,
                      bottom = 1in,
                      left   = 0.8in,
                      right  = 0.8in]
                      {geometry}


%---------------------------------------------------
%  Paragraph formatting
%---------------------------------------------------
\setlength{\parindent}{0pt}
\setlength{\parsep}{10pt}


%---------------------------------------------------
%  List formatting
%---------------------------------------------------
\usepackage{enumitem}
\setlist{noitemsep,topsep=5pt,parsep=0pt,partopsep= 0pt, itemindent=0pt}
%\renewcommand{\labelenumi}{\theenumi )} % redefines enumerate labels to 1), 2) 3) etc.

\newcommand{\benu}{\begin{enumerate}}
\newcommand{\eenu}{\end{enumerate}}

\newcommand{\bitem}{\begin{itemize}}
\newcommand{\eitem}{\end{itemize}}


%---------------------------------------------------
% TiKZ
%---------------------------------------------------
\usepackage{tikz}
\usetikzlibrary{calc,through,intersections, arrows, shapes, matrix, patterns, calligraphy, positioning}
\usetikzlibrary{decorations.pathmorphing, decorations.markings}
\usepackage{tikz-3dplot}
\usetikzlibrary{arrows.meta}
\usetikzlibrary{bending}
\usepackage{pgf, pgfplots}
\usepgflibrary{plotmarks}
\pgfplotsset{compat=1.12}
\tikzset{>=latex}

\newcommand{\btikz}[1][]{\begin{equation*}\begin{tikzpicture}[#1]}
\newcommand{\etikz}{\end{tikzpicture}\end{equation*}}





%---------------------------------------------------
% Colors
%---------------------------------------------------
\definecolor{myblue}{rgb}{0,.25,.6}
\definecolor{light}{gray}{.95}
\definecolor{lines}{RGB}{60, 60, 60}
\definecolor{shadecolor}{rgb}{.95,1,1}



%---------------------------------------------------
% Fonts
%---------------------------------------------------
\usepackage[math]{iwona}
\usepackage{pifont}

% This fixes the missing norm font with Iwona - use \norm{}
\usepackage{mathtools}
\usepackage{xparse}
\DeclarePairedDelimiter\xnorm{\lVert}{\rVert}
\NewDocumentCommand{\norm}{som}
 {\IfBooleanTF{#1}
   {\xnorm*{#3}}
   {\IfNoValueTF{#2}
     {\mathopen{|\mkern-.8mu|}#3\mathclose{|\mkern-.8mu|}}
     {\xnorm[#2]{#3}}%
   }
 }


%---------------------------------------------------
%  Larger fonts 
%---------------------------------------------------
\usepackage{scrextend}
\changefontsizes{15pt}


%---------------------------------------------------
% Section headings
%---------------------------------------------------

\usepackage[explicit]{titlesec}
\titleformat{\section}
  {\bfseries\selectfont}
  {}{0pt}
  {\begin{tcolorbox}[
      enhanced,
      boxrule=0pt,
      arc=0pt,
      outer arc=0pt,
      width=\textwidth,
      top=2pt,
      bottom = 2pt,
      interior code={\fill[red!70!black] (frame.north west) rectangle (frame.south east);},
    ]
    \color{white} MTH 309 \hfill  
    #1
    \end{tcolorbox}}


%---------------------------------------------------
% Page numbering
%---------------------------------------------------
%\numberwithin{page}{section}
%\renewcommand{\thepage}{\thesection-\arabic{page}}

%---------------------------------------------------
% New section macro
%---------------------------------------------------
\newcommand{\lecture}[1]{\newpage\section{#1}}
%\newcommand{\lecture}[1]{\newpage\section{#1}\setcounter{page}{1}}

%---------------------------------------------------
%  Colorboxes
%---------------------------------------------------
\usepackage{tcolorbox}
\tcbuselibrary{skins}

% theorem colorbox
\newenvironment{cbox}[1][]
                            {\begin{tcolorbox}[width=\textwidth,
                                                          fonttitle=\bfseries,
                                                          arc = 0mm,
                                                          toptitle = 0mm,
                                                          beforeafter skip= 20pt,
                                                          bottomtitle = 0mm,
                            			        title = {#1},
                 		 		        enhanced,
                  					boxsep=4pt,
                  					left=8pt,
                  					right=8pt,
                  					top=5pt,
					                 bottom = 5pt,
					                 colframe = red!70!black,
					                 colback  = red!10!white,
                  					]
			  }
			  {\end{tcolorbox}}

% simple text frame
\newenvironment{sframe}
                            {\begin{tcolorbox}[width=\textwidth,
                                                          arc = 0mm,
                                                          toptitle = 0mm,
                                                          beforeafter skip= 10pt,
                                                          bottomtitle = 0mm,
                 		 		        enhanced,
                  					boxsep=4pt,
                  					left=8pt,
                  					right=8pt,
                  					top=0pt,
					                 bottom = 6pt,
					                 colframe = black,
					                 colback  = white,
                  					]
			  }
			  {\end{tcolorbox}}


%---------------------------------------------------
%  Math macros
%---------------------------------------------------


% Matrices - the macro below produces matrices with evenly spaces columns
% with column spacing determined by the width of matrix entries
\usepackage{mathtools,collcell,eqparbox}

\newcounter{BMatrix}

\newlength{\maxwd}
\newcommand{\setmaxwd}[1]{%
  \eqmakebox[BM-\theBMatrix][\BMalign]{$#1$}%
}

\MHInternalSyntaxOn
\newenvironment{BMatrix}[1][r]{
  \def\BMalign{#1}
  \stepcounter{BMatrix}
  \left[
  \MH_let:NwN \@ifnextchar \MH_nospace_ifnextchar:Nnn
  \array{ #1 *{\numexpr\c@MaxMatrixCols-1} {>{\collectcell\setmaxwd}#1<{\endcollectcell}}}
  }{
  \endarray 
  \right]
}
\MHInternalSyntaxOff
\makeatother

\newcommand{\bbm}[1][r]{\begin{BMatrix}[#1]}
\newcommand{\ebm}{\end{BMatrix}}



% Rings and fields

\newcommand{\N}{{\mathbb N}}
\newcommand{\Z}{{\mathbb Z}}
\newcommand{\Q}{{\mathbb Q}}
\newcommand{\R}{{\mathbb R}}
\newcommand{\C}{{\mathbb C}}
\newcommand{\Poly}{{\mathbb P}}


% Vector space bases
\newcommand{\BB}{{\mathcal B}}
\newcommand{\CC}{{\mathcal C}}
\newcommand{\DD}{{\mathcal D}}
\newcommand{\EE}{{\mathcal E}}
\newcommand{\FF}{{\mathcal F}}
\newcommand{\MM}{{\mathcal M}}

% Vectors

\newcommand{\bb}{{\bf b}}
\newcommand{\cc}{{\bf c}}
\newcommand{\dd}{{\bf d}}
\newcommand{\ee}{{\bf e}}
\newcommand{\mm}{{\bf m}}
\newcommand{\nn}{{\bf n}}
\newcommand{\vv}{{\bf v}}
\newcommand{\ww}{{\bf w}}
\newcommand{\uu}{{\bf u}}
\newcommand{\xx}{{\bf x}}
\newcommand{\yy}{{\bf y}}
\newcommand{\zz}{{\bf z}}
\newcommand{\zzero}{{\bf 0}}


% Arrows
\newcommand{\ra}{\rightarrow}
\newcommand{\lra}{\longrightarrow}
\newcommand{\la}{\leftarrow}
\newcommand{\lla}{\longleftarrow}


% Vector spaces
\newcommand{\Span}{\text{Span}}
\newcommand{\Col}{\text{Col}}
\newcommand{\Row}{\text{Row}}
\newcommand{\Nul}{\text{Nul}}
\newcommand{\Ker}{\text{Ker}}
\newcommand{\Img}{\text{Im}}
\newcommand{\rank}{\text{rank}}
\newcommand{\dist}{\text{dist}}
\newcommand{\proj}{\text{proj}}

\DeclarePairedDelimiter{\innprod}{\langle}{\rangle}



\usepackage{multicol}

\begin{document}



\lecture{True or False ? (part 2)}





{\bf Instructions.}

\vskip 2mm

In this problem you will be given a few statements. For each statement you  need to 
decide if it is true or not, and justify your answer. 



\vskip 15mm


{\bf How to justify your answer.}

\vskip 2mm

\textbullet \  In order to show that a statement is false, it suffices  to give a counterexample. 
For example, consider the the statement: 

\begin{center}
\emph{The last digit of every even number is either 2, 4, or 8.}
\end{center}

To show that this statement is false, it is enough to point out that, for example, 10 is an even number, 
but its last digit is 0. 

\vskip 2mm

\textbullet \  In order to show that a statement is true, you need to 
provide a reasoning explaining why it is true in all instances. Giving one example when it is true will 
not suffice, since the statement may not work in some other cases.  For example, consider the the statement: 

\begin{center}
\emph{If $n$ is an even number then $n+2$ is also an even number.}
\end{center}

You can justify that this is true as follows. Even numbers are integers which are multiples of $2$. 
If $n$ is even then $n = 2m$ for some integer $m$. Then $n+2 = 2m+2 = 2(m+1)$, 
which shows that $n+2$ is even.  


\hfill

\begin{cbox}[Note]
This problem will not be collected or graded. However, problems of this type will appear on  exams
in this course. Sample solutions are provided at the end. 

\end{cbox}


%%%%%%%%%%%
\newpage
%%%%%%%%%%%

{\bf For each of the statements given below decide if it is true or false. If you decide that it is true, justify your answer. 
If you think it is false give a counterexample. }

\vskip 10mm

\benu
\item[\bf a)] If $\vv_{1}, \vv_{2}, \vv_{3}$ are vectors in $\R^{3}$ and $\vv_{1} \in \Span(\vv_{2}, \vv_{3})$ then 
the set $\{\vv_{1}, \vv_{2}, \vv_{3} \}$ is linearly dependent. 

\vskip 10mm

\item[\bf b)] If $\vv_{1}, \vv_{2}, \vv_{3}$ are vectors in $\R^{3}$ and the set $\{\vv_{1}, \vv_{2}\}$ is linearly independent, 
then the set $\{\vv_{1}, \vv_{2}, \vv_{3}\}$ is also linearly independent.  


\vskip 10mm

\item[\bf c)]  If $A$ is a $2\times 3$ matrix, and $\vv$ is a vector in the column space $\Col(A)$ then $\vv\in \R^{3}$.

\vskip 10mm

\item[\bf d)]  If $A$ is a matrix, $\Nul(A)$ is the null space of $A$, and $\vv$ is a non-zero vector such that 
$\vv\in\Nul(A)$, then $\Nul(A)$ must contain infinitely many vectors. 


\eenu


%%%%%%%%%%%
\newpage
%%%%%%%%%%%

{\bf Here are solutions to the questions from the previous page. You should try to answer all 
questions by yourself before reading these solutions.}

\vskip 5mm 

{\small
\benu

\item[\bf a)] TRUE. If $\vv_{1} \in \Span(\vv_{2}, \vv_{3})$ then  $\vv_{1}  = c_{2}\vv_{2} + c_{3}\vv_{3}$, 
for some scalars $c_{1}, c_{2}$, or equivalently
$$\vv_{1}  - c_{2}\vv_{2} - c_{3}\vv_{3} = \zzero$$
This means that the vector equation $x_{1}\vv_{1}  + x_{2}\vv_{2} + x_{3}\vv_{3} = \zzero$ has a non-trivial 
solution $x_{1} = 1$, $x_{2} = -c_{2}$, $x_{3} = -c_{3}$. Thus the set $\{\vv_{1}, \vv_{2}, \vv_{3} \}$ is linearly dependent. 

\vskip 5mm


\item[\bf b)] FALSE. For example, take:


$$
\vv_{1} = 
\bbm
1 \\ 0 \\ 0
\ebm, 
\vv_{2} = 
\bbm
0 \\ 1 \\ 0
\ebm, 
\vv_{3} = 
\bbm
1 \\ 1 \\ 0
\ebm, 
$$
The set $\{\vv_{1}, \vv_{2}\}$ is linearly independent (since the equation $x_{1}\vv_{1} + x_{2}\vv_{2} = \zzero$
has only one, trivial solution). On the other hand the set $\{\vv_{1}, \vv_{2}, \vv_{3}\}$ is  linearly dependent because the 
equation $x_{1}\vv_{1} + x_{2}\vv_{2}  + x_{3}\vv_{3}= \zzero$ has non-trivial solutions, e.g.
$x_{1} =1$, $x_{2} = 1$, $x_{3} = -1$.


\vskip 5mm

\item[\bf c)]  FALSE. The column space of a matrix $A$ consists of linear combinations of columns of $A$. Since 
every column of a $2\times 3$ matrix has 2 entries, thus every vector of $\Col(A)$ is a vector in $\R^{2}$, and 
not in $\R^{3}$.  


\vskip 5mm

\item[\bf d)] TRUE. The null space of $A$ consists of all vectors which are solutions of the matrix equation $A\xx = \zzero$. 
This equation always has the trivial solution $\xx = \zzero$. If $\vv$ is a non-zero vector in $\Nul(A)$, then $\xx = \vv$
is another solution of this equation. This implies that $A\xx = \zzero$ has infinitely many solutions, and so $\Nul(A)$
consists of infinitely many vectors. 





\eenu
}



\end{document}